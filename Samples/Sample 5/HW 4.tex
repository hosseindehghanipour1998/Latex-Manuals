% ====================== Packages =============================
\documentclass[a4paper,12pt]{article}			   
\usepackage[left=2cm, right=2cm, top=2.5cm, bottom=2.5cm]{geometry}
\usepackage{fancyhdr , lipsum}
\usepackage{mathptmx}
\usepackage{anyfontsize}
\usepackage{t1enc}
\usepackage{csquotes}
\usepackage{enumitem}
\usepackage{xcolor}
\usepackage[linktocpage]{hyperref}
\usepackage{graphicx}
\usepackage{float}
\usepackage{subcaption}
\usepackage[section]{placeins} 
\usepackage[T1]{fontenc}
\usepackage{fontspec}
\usepackage{tcolorbox}
\usepackage[english]{babel}
\usepackage[export]{adjustbox}
\usepackage{parskip}
\usepackage{siunitx}
\usepackage{amsmath}
\usepackage{mathtools}
\usepackage{tikz}
\def\checkmark{\tikz\fill[scale=0.4](0,.35) -- (.25,0) -- (1,.7) -- (.25,.15) -- cycle;} 
\usepackage{setspace}
% Header and Footer
\usepackage{lipsum}%% a garbage package you don't need except to create examples.
\usepackage{fancyhdr}
\pagestyle{fancy}
\lhead{Linear Algebra HW 4}
\rhead{ Page :  \thepage}
\cfoot{Hossein Dehghanipour - 9532250}
% End ofHeader and Footer
\renewcommand{\headrulewidth}{0.4pt}
\renewcommand{\footrulewidth}{0.4pt}
\newcommand{\thedate}{\today}
\setlength{\parindent}{0pt}
\usepackage{atbegshi}% http://ctan.org/pkg/atbegshi
\AtBeginDocument{\AtBeginShipoutNext{\AtBeginShipoutDiscard}}
% ====================== Begin Doc =============================
\begin{document}

    \begin{titlepage}
        \begin{center}
            \title{\Large{\textbf{Linear Algebra - HW 4}}}
            \author{Hossein Dehghanipour - 9532250}
            \date{\today}
        \end{center}
    \end{titlepage}

    \maketitle
\begin{doublespace}
\begin{flushleft}
\begin{doublespace}
	Teacher :  Dr.Mehrzad Bighash
\end{doublespace} 
\begin{doublespace}
	TA : Ms.F.Lotfi
\end{doublespace} 
\end{flushleft}
\end{doublespace} 

\newpage
\setmainfont{Times New Roman}
\pagenumbering{roman}
\tableofcontents
\thispagestyle{empty}
\newpage
\pagenumbering{arabic}

% *************** Questions Section ******************

\fontsize{15pt}{15pt}
\newcommand{\nnl}{\newline \noindent}
\newcommand{\ptx}{ P_{tx} }
\newcommand{\prx}{ P_{rx} }
%===================================
\section {Question 1}

\begin{figure}[h!]
	\centering
	\includegraphics*[height=3cm]{Q1}
\end{figure}

\begin{doublespace}
$Ker(T(v)) = \{ x \mid x \in V , T(x) = 0\}  $ \\
$T : v  \Rightarrow W $\\
In order to prove that a Kernel is a subspace, we should prove two ,main features of a kernel :
\begin{itemize}
  \item It's closed under vector addition.
  \item It's closed under multiplication.
\end{itemize}

Now proving the first feature :\\
Suppose : $x , y \in Ker(T) $ this means that : $T(x) = 0$   \&   $T(y) = 0 $\\
can we imply that $ (x + y) \in Ker(T) $ ??\\
$ T( x + y ) = T(x) + T(y) = 0 + 0 = 0 \in Ker(T) \Rightarrow x + y \in Ker(T)$    \checkmark \\
Now Proving the second feature :\\
Assume: $x \in K(T) $ Then $\Rightarrow T(\alpha x ) = \alpha T(x) = \alpha 0 = 0 \in K(T) \Rightarrow $ The second feature is also proved \checkmark \\
Now We can say that the Kernel of a Linear Transformation is a subspace.
\end{doublespace}
\line(1,0){450}
\newpage
%===================================
\section {Question 2 }
\begin{figure}[h!]
	\centering
	\includegraphics*[height=2cm]{Q2}
\end{figure}

\begin{doublespace}
Assume : $\alpha = VV^T $ and $\beta =V^TV $ \\
$\beta ^T = \alpha $ and $\alpha^T = \beta$\\
$(x/y)^T = y^T/x^T$\\ 
$C_v = I - \frac{\alpha}{\beta} \Rightarrow C_v ^T = I^T - \frac{(\beta)^T}{(\alpha)^T} \Rightarrow C_v^T = I^T - \frac{(\alpha)}{(\beta)} = I - \frac{\alpha}{\beta}$\\
As we can see $C_v^T$ is equal with $C_v$ which means : $C_v^T = C_v \Rightarrow C_v$ is a parallel matrix. 
\end{doublespace}
\line(1,0){450}
\newpage
%===================================

%===================================
\section {Question 3 }
\begin{figure}[h!]
	\centering
	\includegraphics*[height=4cm]{Q3}
\end{figure}

\begin{doublespace}
$X = [  x , y ]$\\
$S(X) = \begin{pmatrix} x-y\\x+y \end{pmatrix}$\\
$T(X) = \begin{pmatrix} x+y\\x-y \end{pmatrix}$\\
\end{doublespace} 

\begin{doublespace}
$ToS : \begin{pmatrix} (x-y)+(x+y)\\ (x-y)-(x+y) \end{pmatrix} = \begin{pmatrix} 2x\\ -2y \end{pmatrix}$
\end{doublespace} 

\begin{doublespace}
$SoT : \begin{pmatrix} (x+y)-(x-y)\\ (x+y)+(x-y) \end{pmatrix} = \begin{pmatrix} 2y\\ 2x \end{pmatrix}$
\end{doublespace} 

\line(1,0){450}
\newpage
%===================================
\section {Question 4 }
\begin{figure}[h!]
	\centering
	\includegraphics*[height=2cm]{Q4}
\end{figure}

\noindent
\begin{doublespace}
$T(i_1) = i_1 - 3i_2  \Rightarrow T(i_1) =  \binom{1}{-3} $

\noindent $ T(i_2) = i_2  \Rightarrow T(i_2) =  \binom{0}{1}$

$A = [ T(i_1)$   $  T(i_2) ]  = \begin{pmatrix} 1 & 0\\-3 & 1 \end{pmatrix} $ \\
testing : \\
$T(x) = Ax \Rightarrow T(i_1) = \begin{pmatrix} 1 & 0\\-3 & 1 \end{pmatrix} \begin{pmatrix} 1 \\0 \end{pmatrix} = \begin{pmatrix} 1 \\-3 \end{pmatrix}  $ \checkmark \\
\end{doublespace}

\begin{doublespace}
$T(x) = Ax \Rightarrow T(i_2) = \begin{pmatrix} 1 & 0\\-3 & 1 \end{pmatrix} \begin{pmatrix} 1 \\0 \end{pmatrix} = \begin{pmatrix} 0 \\1 \end{pmatrix}  $ \checkmark \\
\end{doublespace}
\line(1,0){450}
\newpage
%===================================
End of this HomeWork.\\
This file is written in \LaTeX
% ====================== End Doc =============================
\end{document}