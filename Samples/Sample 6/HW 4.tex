% ====================== Packages =============================
\documentclass[a4paper,12pt]{article}			   
\usepackage[left=2cm, right=2cm, top=2.5cm, bottom=2.5cm]{geometry}
\usepackage{fancyhdr , lipsum}
\usepackage{mathptmx}
\usepackage{anyfontsize}
\usepackage{t1enc}
\usepackage{csquotes}
\usepackage{enumitem}
\usepackage{xcolor}
\usepackage[linktocpage]{hyperref}
\usepackage{graphicx}
\usepackage{float}
\usepackage{cancel}
\usepackage{subcaption}
\usepackage[section]{placeins} 
\usepackage[T1]{fontenc}
\usepackage{fontspec}
\usepackage{tcolorbox}
\usepackage[english]{babel}
\usepackage[export]{adjustbox}
\usepackage{parskip}
\usepackage{siunitx}
\usepackage{amsmath}
\usepackage{mathtools}
\usepackage{tikz}
\usepackage{soul}
\def\checkmark{\tikz\fill[scale=0.4](0,.35) -- (.25,0) -- (1,.7) -- (.25,.15) -- cycle;} 
\usepackage{setspace}
% Header and Footer
\usepackage{lipsum}%% a garbage package you don't need except to create examples.
\usepackage{fancyhdr}
\pagestyle{fancy}
\lhead{Linear Algebra HW 5}
\rhead{ Page :  \thepage}
\cfoot{Hossein Dehghanipour - 9532250}
% End ofHeader and Footer
\renewcommand{\headrulewidth}{0.4pt}
\renewcommand{\footrulewidth}{0.4pt}
\newcommand{\thedate}{\today}
\setlength{\parindent}{0pt}
\usepackage{atbegshi}% http://ctan.org/pkg/atbegshi
\AtBeginDocument{\AtBeginShipoutNext{\AtBeginShipoutDiscard}}


%====================== Border ================================

% ====================== Begin Doc =============================
\begin{document}

    \begin{titlepage}
        \begin{center}
            \title{\Large{\textbf{Linear Algebra - HW 5}}}
            \author{Hossein Dehghanipour - 9532250}
            \date{\today}
        \end{center}
    \end{titlepage}

    \maketitle
\begin{doublespace}
\begin{flushleft}
\begin{doublespace}
	Teacher :  Dr.Mehrzad Bighash
\end{doublespace} 
\begin{doublespace}
	TA : Ms.F.Lotfi
\end{doublespace} 
\end{flushleft}
\end{doublespace} 

\newpage
\setmainfont{Times New Roman}
\pagenumbering{roman}
\tableofcontents
\thispagestyle{empty}
\newpage
\pagenumbering{arabic}

% *************** Questions Section ******************

\fontsize{15pt}{15pt}
\newcommand{\nnl}{\newline \noindent}
\newcommand{\ptx}{ P_{tx} }
\newcommand{\prx}{ P_{rx} }
%===================================
\section {Question 1}

\begin{figure}[h!]
	\centering
	\includegraphics*[height=3cm]{Q1}
\end{figure}

\begin{figure}[h!]
	\centering
	\includegraphics*[height=8cm]{Pic1}
\end{figure}

\begin{align*}
	&S = \pi r^2 = 16\pi \\
	&S^\prime = S * \left\| det \begin{pmatrix} a & b\\c & d \end{pmatrix} \right\|  \Rightarrow \\
	&S^\prime = S *\left\| ad - bc \right\| \Rightarrow 16\pi \left\| ad - bc \right\|
\end{align*}
\line(1,0){450}
\newpage
%===================================
\section {Question 2 }
\begin{figure}[h!]
	\centering
	\includegraphics*[height=3cm]{Q2}
\end{figure}

\begin{doublespace}
	As we know , $ I_n = \begin{pmatrix} 1 & 0 & 0 & 0\\0 & 1 & 0 & 0\\0 & 0 & \ddots & 0\\0 & 0 & 0 & 1 \end{pmatrix}_{n\times n}$ \\
	And we also know that $A \otimes I_n = \begin{pmatrix} A & 0 & 0 & 0\\0 & A & 0 & 0\\0 & 0 & \ddots & 0\\0 & 0 & 0 & A \end{pmatrix}_{n\times n}$\\
	We also know as a fact that $ det\begin{pmatrix} A & 0\\0 & D \end{pmatrix} = det(A)det(D) \Rightarrow $ so \\
	$det\begin{pmatrix} A & 0 & 0 & 0\\0 & A & 0 & 0\\0 & 0 & \ddots & 0\\0 & 0 & 0 & A \end{pmatrix}_{n\times n} =\underbrace{det(A)*det(A)*....*det(A)}_{\text{n times}} = det(A)^n $\\
\end{doublespace}

\line(1,0){450}
\newpage
%===================================

%===================================
\section {Question 3 }

\begin{figure}[h!]
	\centering
	\includegraphics*[height=2cm]{Q3}
\end{figure}

\begin{doublespace}
	\begin{align*}
		&det(AB^2) = 72 \Rightarrow det(A)det(B)^2 = 72 \\
		&det(A^2B^2) = 144 \Rightarrow det(A)^2 * det(B)^2 = 144\\
		&det(A) = ? , det(2A) = ? , det(AB) = ? \\
		&\frac{det(A^2B^2)}{det(AB^2)} = \frac{144}{72} \Rightarrow  \frac{ det(A) \cancel{det(A)}  \cancel{ det(B)^2}}{\cancel{det(A)} \cancel{det(B)^2}} \Rightarrow \textcolor{blue}{ det(A) = 2}  \\
		& det(A)det(B)^2 = 72 \Rightarrow 2 * det(B)^2 = 72 \Rightarrow \textcolor{blue}{ det(B) = 6} \\
		&det(2A) = 2det(A) = 2 * 2 = 4   \Rightarrow \textcolor{blue}{ det(2A) = 4 }\\
		&det(AB) = det(A) det(B) = 6 * 2 = 12 \Rightarrow \textcolor{blue}{det(AB) = 12 }
	\end{align*}
\end{doublespace}
\line(1,0){450}
\newpage
%===================================
\section {Question 4 }

\begin{figure}[h!]
	\centering
	\includegraphics*[height=5cm]{Q4}
\end{figure}

\subsection{Part I}
\begin{doublespace}
I would like to calculate the determinant using the second row due to having the most number of zeros which makes my calculations a lot easier.
	\begin{align*}
		&A = \begin{pmatrix} 0 & 5 & 1 & 4\\ 0 & 1 & 0 & 0\\ 8 & 1 & 7 & 5\\ 2 & 1 & 9 & 4\\ \end{pmatrix}\\
		& det(A) = (-1) det(A_{22}) + 0 = -det(A_{22}) \\
		& A^\prime = A_{22} = \begin{pmatrix} 0 & 1 & 4\\ 8  & 7 & 5\\ 2 & 9 & 4\\ \end{pmatrix} \\
		&det(A_{22}) = 0 * \dots -(1) det(A^\prime_{12}) + 4*det(A^\prime_{13}) \Rightarrow \\
		& = (-1)\begin{vmatrix}
			8 & 5  \\ 
			2 & 4 \\ 
			\end{vmatrix}  
		+ 
		(4)\begin{vmatrix}
			8 & 7 \\ 
			2 & 9  \\ 
			\end{vmatrix}  
		= (32 - 10 ) * (-1) + 4(72-14) = -22 + 4(58) = 232 - 22 = 210 
	\end{align*}
\end{doublespace}
\line(1,0){450}

\subsection{Part II}
\begin{doublespace}
I would like to calculate the determinant using the second row due to having the most number of zeros which makes my calculations a lot easier.
	\begin{align*}
		&B = \begin{pmatrix} 4 & 2 & 7 & 0\\ 3 & 6 & 9 & 4\\ 1 &4 & 3 & 9\\ 4 & 0 & 1 & 1\\ \end{pmatrix}\\
		& det(B) = (4) det(B_{11}) -  (3) det(B_{12}) +  (7) det(B_{13})  \\
		& \textcolor{red}{det(B_{11}) = ? } \Rightarrow  B_{11} = \begin{pmatrix} 6 & 9 & 4\\ 4  & 3 & 9\\ 0 & 1 & 1\\ \end{pmatrix} \Rightarrow \\%% Det(11) 
		& = (6)\begin{vmatrix} 3 & 9  \\ 1 & 1 \\ \end{vmatrix}  + (-4)\begin{vmatrix} 9 & 4 \\ 1 & 1  \\ \end{vmatrix} = (6*-6 ) - 4*5 =  -36 -20 = -56 \Rightarrow \\
		& \textcolor{blue}{det(B_{11}) =  -56 } \\
		& \textcolor{red}{det(B_{12}) = ? } \Rightarrow  B_{12} = \begin{pmatrix} 3 & 9 & 4\\ 1  & 3 & 9\\ 4 & 1 & 1\\ \end{pmatrix} \Rightarrow \\%% Det(12) 
		& = (3)\begin{vmatrix} 3 & 9  \\ 1 & 1 \\ \end{vmatrix}  + (-9)\begin{vmatrix} 1 & 9 \\ 4 & 1  \\ \end{vmatrix} + (4)\begin{vmatrix} 1 & 3 \\ 4 & 1  \\ \end{vmatrix} = 3(-6) -9(-35) + 4(-11) = 253 \Rightarrow \\
		& \textcolor{blue}{det(B_{12}) = 253 } \\
		& \textcolor{red}{det(B_{13}) = ? } \Rightarrow  B_{13} = \begin{pmatrix} 3 & 6 & 4\\ 1  & 4 & 9\\ 4 & 0 & 1\\ \end{pmatrix} \Rightarrow \\%% Det(13) 
		& = (4)\begin{vmatrix} 6 & 4  \\ 4 & 9 \\ \end{vmatrix}  + (1)\begin{vmatrix}3 & 6 \\ 1 & 4  \\ \end{vmatrix} = 4(56-16)  + (12-6) = 158 \Rightarrow \\
		& \textcolor{blue}{det(B_{13}) = 158 } \\
		&det(B) = 4(-56) -3(253) +7(158) = 123 \\
	\end{align*}
\end{doublespace}
\line(1,0){450}
\newpage
%===================================
\section {Question 5 }

\begin{figure}[h!]
	\centering
	\includegraphics*[height=1.5cm]{Q5}
\end{figure}

\begin{doublespace}
Prove that $det(A^{adj}) = det(A)^{n-1} $\\
	\begin{align*}
		&A^{-1} = \frac{A^{adj}}{det(A)} \Rightarrow \\
		 &A^{adj} = A^{-1} *  det(A) \xrightarrow{\text{*A from left}} \\
		&A * A^{adj} = A*A^{-1} *  det(A) \Rightarrow \\
		&A * A^{adj} = I_n * det(A) \xrightarrow{\text{det() from both sides}} \\
		&det(A * A^{adj}) =det( I_n * det(A))\\
	\end{align*}
We know that $ det(\alpha I_n) = \alpha^n $\\
	\begin{align*}
	&det(A * A^{adj}) =det( I_n * det(A)) \Rightarrow \\
	&det(A) * det(A^{adj}) =  det(A)^n \Rightarrow \\
	&det(A^{adj}) = det(A)^{n-1}
	\end{align*}
\end{doublespace}
\line(1,0){450}
\newpage
%===================================
End of this HomeWork.\\
This file is written in \LaTeX
% ====================== End Doc =============================
\end{document}