%%%% https://www.overleaf.com/learn/latex/Code_Highlighting_with_minted
\documentclass[10pt , a4paper]{article}						   
\usepackage[left=2cm, right=2cm, top=2.5cm, bottom=2.5cm]{geometry}
\usepackage{fancyhdr , lipsum}
\usepackage{mathptmx}
\usepackage{anyfontsize}
\usepackage{t1enc}
\usepackage{csquotes}
\usepackage{blindtext}
\usepackage{enumitem}
\usepackage{xcolor}
\usepackage[linktocpage]{hyperref}
\usepackage{graphicx}
\usepackage{float}
\usepackage{subcaption}
\usepackage[section]{placeins} 
\usepackage[T1]{fontenc}
\usepackage{fontspec}
\usepackage{tcolorbox}
\usepackage[english]{babel}
\usepackage[export]{adjustbox}

%%%%%%%%%%%%%%%%%%%%%%%%%%%%%%
\usepackage[T1]{fontenc}
\usepackage{inconsolata}

\usepackage{color}

\definecolor{pblue}{rgb}{0.13,0.13,1}
\definecolor{pgreen}{rgb}{0,0.5,0}
\definecolor{pred}{rgb}{0.9,0,0}
\definecolor{pgrey}{rgb}{0.46,0.45,0.48}

\usepackage{listings}
\lstset{language=Java,
  showspaces=false,
  showtabs=false,
  breaklines=true,
  showstringspaces=false,
  breakatwhitespace=true,
  commentstyle=\color{pgreen},
  keywordstyle=\color{pblue},
  stringstyle=\color{pred},
  basicstyle=\ttfamily,
  moredelim=[il][\textcolor{pgrey}]{$$},
  moredelim=[is][\textcolor{pgrey}]{\%\%}{\%\%}
}

%%%%%%%%%%%%%%%%%%%%%%%%%%%%%%%%%
\newcommand{\Topic}{Chain of Responsibility Design Pattern}
%%%%%%%%%%%%%%%% Header/Footer %%%%%%%%%
\renewcommand{\footrulewidth}{0.4pt}% default is 0pt
\pagestyle{fancy}
\fancyhf{}
\chead{\textbf{\Topic}}
\rhead{Derek Banas - Youtube - 2012}
\lhead{Java Design Patterns}
\cfoot{Page \thepage}
\lfoot{Hossein Dehghanipour}
\rfoot{\today}
%%%%%%%%%%%%%%%%%%%%%%%%%%%%%%%%
\begin{document}

\huge{\textbf{\Topic}}
\large
\\[7pt]
%%%%% Description
\begin{itemize}
   \item This pattern sends data to an object and if that object can't use it, it sends it to any number of other objects that may be able to use it.  \\[3pt]
	\begin{itemize}
		 \item Create 4 objects that can either add, subtract, multiply, or divide. \\[3pt]
		 \item Send 2 numbers and a command and allow these 4 objects to decide which can handle the requested calculation. \\[3pt]
	\end{itemize}
  % \item . \\[3pt]
  % \item \\[3pt]
   %\item \\[3pt]

\end{itemize}


\newpage

%%%%%%%%%%

\small
\begin{lstlisting}[language=java]

//===================================
// AddNumbers.java
public class AddNumbers implements Chain{

	private  Chain nextInChain;
	
	// Defines the next Object to receive the
	// data if this one can't use it
	
	public void setNextChain(Chain nextChain) {
		
		nextInChain = nextChain;
		
	}

	// Tries to calculate the data, or passes it
	// to the Object defined in method setNextChain()
	
	public void calculate(Numbers request) {
		
		if(request.getCalcWanted() == "add"){
			
			System.out.print(request.getNumber1() + " + " + request.getNumber2() + " = "+
					(request.getNumber1()+request.getNumber2()));
			
		} else {
			
			nextInChain.calculate(request);
			
		}	
	}
}
 
//===================================
// Chain.java
// The chain of responsibility pattern has a 
// group of objects that are expected to between
// them be able to solve a problem. 
// If the first Object can't solve it, it passes
// the data to the next Object in the chain

public interface Chain {

	// Defines the next Object to receive the data
	// if this Object can't process it
	
	public void setNextChain(Chain nextChain);
	
	// Either solves the problem or passes the data
	// to the next Object in the chain
	
	public void calculate(Numbers request);
	
}
 
//===================================
// DivideNumbers.java
public class DivideNumbers implements Chain{

	private  Chain nextInChain;
	
	@Override
	public void setNextChain(Chain nextChain) {
		
		nextInChain = nextChain;
		
	}

	@Override
	public void calculate(Numbers request) {
		
		if(request.getCalcWanted() == "div"){
			
			System.out.print(request.getNumber1() + " / " + request.getNumber2() + " = "+
					(request.getNumber1()/request.getNumber2()));
			
		} else {
			
			System.out.print("Only works for add, sub, mult, and div");
			
		}
	}
}
 
//===================================
// MultNumbers.java
public class MultNumbers implements Chain{

	private  Chain nextInChain;
	
	@Override
	public void setNextChain(Chain nextChain) {
		
		nextInChain = nextChain;
		
	}

	@Override
	public void calculate(Numbers request) {
		
		if(request.getCalcWanted() == "mult"){
			
			System.out.print(request.getNumber1() + " * " + request.getNumber2() + " = "+
					(request.getNumber1()*request.getNumber2()));
			
		} else {
			
			nextInChain.calculate(request);
			
		}	
	}
}
 
//===================================
// Numbers.java
// This object will contain 2 numbers and a
// calculation to perform in the form of a String

public class Numbers {

	private int number1;
	private int number2;
	
	private String calculationWanted;
	
	public Numbers(int newNumber1, int newNumber2, String calcWanted){
		
		number1 = newNumber1;
		number2 = newNumber2;
		calculationWanted = calcWanted;
		
	}
	
	public int getNumber1(){ return number1; }
	public int getNumber2(){ return number2; }
	public String getCalcWanted(){ return calculationWanted; }
	
}
 
//===================================
// SubtractNumbers.java
public class SubtractNumbers implements Chain{

	private  Chain nextInChain;
	
	@Override
	public void setNextChain(Chain nextChain) {
		
		nextInChain = nextChain;
		
	}

	@Override
	public void calculate(Numbers request) {
		
		if(request.getCalcWanted() == "sub"){
			
			System.out.print(request.getNumber1() + " - " + request.getNumber2() + " = "+
					(request.getNumber1()-request.getNumber2()));
			
		} else {
			
			nextInChain.calculate(request);
			
		}
		
	}
}
 
//===================================
// TestCalcChain.java
public class TestCalcChain {
	
	public static void main(String[] args){
		
		// Here I define all of the objects in the chain
		
		Chain chainCalc1 = new AddNumbers();
		Chain chainCalc2 = new SubtractNumbers();
		Chain chainCalc3 = new MultNumbers();
		Chain chainCalc4 = new DivideNumbers();
		
		// Here I tell each object where to forward the
		// data if it can't process the request
		
		chainCalc1.setNextChain(chainCalc2);
		chainCalc2.setNextChain(chainCalc3);
		chainCalc3.setNextChain(chainCalc4);
		
		// Define the data in the Numbers Object
		// and send it to the first Object in the chain
		
		Numbers request = new Numbers(4,2,"add");
		
		chainCalc1.calculate(request);
		
	}
}
 

\end{lstlisting}
\end{document}
