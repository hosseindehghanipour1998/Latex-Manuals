%%%%%%%%%%%%%%%%%%%%%%%%%%%%%%Packages%%%%%%%%%%%%%%%%%%%%%%%%%%%%%%
\documentclass[12pt,a4paper]{article}							   %
\usepackage[left=2cm, right=2cm, top=2cm, bottom=2cm]{geometry}	   %
%\usepackage{booktaps}											   %
\usepackage[usenames, dvipsnames]{color}						   %
\usepackage{color}	
\usepackage{xcolor}

% Header and Footer
\usepackage{fancyhdr}
\pagestyle{fancy}
\lhead{\lr{Embedded Systems :  HW 2.3}}
\rhead{ \lr{Page :  \thepage}}
\cfoot{\lr{Hossein Dehghanipour - 9532250}}
\renewcommand{\headrulewidth}{0.4pt}% Default \headrulewidth is 0.4pt
\renewcommand{\footrulewidth}{0.5pt}% Default \footrulewidth is 0pt
% End ofHeader and Footer
\usepackage{amsmath}% http://ctan.org/pkg/amsmath
\newcommand{\thedate}{\today}											   %
\newcommand{\dbar}{d\hspace*{-0.08em}\bar{}\hspace*{0.1em}}		   %
\usepackage{amsthm,amssymb,amsmath}	  							   %
\usepackage{hyperref}
\usepackage{MnSymbol,wasysym}
\hypersetup{
    colorlinks=true,
    linkcolor=cyan,
    filecolor=magenta,      
    urlcolor=blue,
}											   %
\usepackage{graphicx,wrapfig}			   					       %
\usepackage{xepersian}			   								   %
\settextfont{B Nazanin}											   %
\defpersianfont\titr[Scale=1]{XB Zar}	  	
				   
%%%%%%%%%%%%%%%%%%%%%%%%%%%%%%Packages%%%%%%%%%%%%%%%%%%%%%%%%%%%%%%
\begin{document}\thispagestyle{empty}

\begin{center}
	\section*{
	 \Huge 
	به نام خدا
	}
	\begin{RTL}
		\large  اصول طراحی سیستم های نهفته\\
		\large همورک دوم \\
		\today
	\\[2\baselineskip]
	\end{RTL}
\end{center}



\begin{flushright}
	\begin{RTL} 
استاد درس :  \href{https://www.linkedin.com/in/mohsen-raji-97974452/}{دکتر محسن راجی}\\
 نویسنده : حسین دهقانی پور -   9532250

	
	\end{RTL}
\end{flushright}


\newpage
\tableofcontents
\begin{flushleft}

\lr{This document is written by \LaTeX.}

\end{flushleft}
\newpage

%%%%%%%%%%%%%%%%%%%%%%%%%%%%%%%%%%%%%%%%%%%%%%%%%%%%%%%%%%%%%%%%%%%%%%%%%%%%%%%%%%%
\section{سوال اول}
	\begin{RTL}
		\textcolor{blue}{
			\lr{Statechart}
 زیر را در نظر بگیرید.اگر ترتیب ورودی ها به صورت زیر (از چپ به راست) باشد، به ترتیب چه حالاتی طی می شوند؟ \\
		}
		\begin{figure}[h!]
		\centering
		\includegraphics*[height=5cm]{Q1}
		\end{figure}
\\	\begin{flushleft}	
	\lr{Solution : We are keeping history so we have $\longrightarrow$ B,C,D,Z,D}
	\end{flushleft}
	
	\end{RTL}
\newpage

%%%%%%%%%%%%%%%%%%%%%%%%%%%%%%%%%%%%%%%%%%%%%%%%%%%%%%%%%%%%%%%%%%%%%%%%%%%%%%%%%%%%
\section{سوال دوم }
	\begin{RTL}
			\textcolor{blue}{
1.	می خواهیم توصیفی برای کنترل کننده یک مایکروفر ارائه کنیم. این مایکروفر خصوصیات زیر را دارد:\\
-	به کمک این مایکروفر می توان غذا را از زیر حرارت داد
\lr{ (cook) }
یا از بالا
\lr{ (grill)} \\
-	برای حالت
\lr{ cook}
 می توان زمان های مختلف
\lr{$ 1، 5، 10، 20 $}
دقیقه را برای پخت غذا انتخاب نمود\\
-	برای حالت 
\lr{grill }
هم دو شدت مختلف
\lr{ low}
و
\lr{ high}
 برای حرارت در نظر گرفته شده است،\\
-	برای مایکروفر یک صفحه ارتباط با کاربر در نظر گرفته شده است که تمامی اطلاعات لازم برای کار مایکروفر از آن دریافت می شود. \\
مایکروفر به این صورت عمل می کند که: \\
-	ابتدا که مواد غذایی در مایکروفر قرار گرفت، بایستی تنظیمات لازم انجام شود، به طور پیش فرض حالت
 \lr{cook}
 و زمان یک دقیقه در نظر گرفته می شود، \\
-	پس از تنظیمات کلید
 \lr{Start}
 زده می شود و مایکرو فر شروع به کار می کند (البته چک می کند که حتما درب مایکروفر بسته باشد) و پس از سپری شدن زمان تنظیم شده، با به صدا در آمدن
\lr{ beep}
 به کاربر اعلام می کند که کار مدنظر به اتمام رسیده است. \\
-	در صورتی که پیش از به پایان رسیدن زمان تنظیم شده، کاربر از ادامه کار منصرف شود، می توان به فشردن کلید
\lr{ Cancel}
 کار را متوقف کند، در این صورت با به صدا در آمدن
\lr{ beep}
 به کاربر اعلام می کند که کار متوقف شده است، \\
-	اگر کاربر درب مایکروفر را قبل از به اتمام رسیدن کار باز کند، باز هم مایکروفر کار را متوقف کرده و صدای
\lr{ beep}
 را پخش می کند. \\
با استفاده از زبان
\lr{ Statechart}
 کنترل کننده مایکروفر را توصیف کنید. دقت داشته باشید که در صورت نیاز، توصیف بایستی با استفاده از ابرحالت ها و سلسله مراتب مناسب انجام شود\\
  \\
			}

	\end{RTL}
		\begin{figure}[h!]
		\centering
		\includegraphics*[height=10cm]{Microwave}
		\end{figure}
		
		\begin{figure}[h!]
		\centering
		\includegraphics*[height=10cm]{Cooker}
		\end{figure}
		
		\begin{figure}[h!]
		\centering
		\includegraphics*[height=10cm]{Griller}
		\end{figure}
\newpage
%%%%%%%%%%%%%%%%%%%%%%%%%%%%%%%%%%%%%%%%%%%%%%%%%%%%%%%%%%%%%%%%%%%%%%%%%%%%%%%%%%%%
\end{document}

