% ====================== Packages =============================
\documentclass[a4paper,12pt]{article}			   
\usepackage[left=2cm, right=2cm, top=2.5cm, bottom=2.5cm]{geometry}
\usepackage{fancyhdr , lipsum}
\usepackage{mathptmx}
\usepackage{anyfontsize}
\usepackage{t1enc}
\usepackage{csquotes}
\usepackage{enumitem}
\usepackage{xcolor}
\usepackage[linktocpage]{hyperref}
\usepackage{graphicx}
\usepackage{float}
\usepackage{cancel}
\usepackage{subcaption}
\usepackage[section]{placeins} 
\usepackage[T1]{fontenc}
\usepackage{fontspec}
\usepackage{tcolorbox}
\usepackage[english]{babel}
\usepackage[export]{adjustbox}
\usepackage{parskip}
\usepackage{siunitx}
\usepackage{amsmath}
\usepackage{mathtools}
\usepackage{tikz}
\usepackage{soul}
\usepackage{unicode-math}
\usepackage{verbatim} % a package used for multiline commenting
\def\checkmark{\tikz\fill[scale=0.4](0,.35) -- (.25,0) -- (1,.7) -- (.25,.15) -- cycle;} 
\usepackage{setspace}
% Header and Footer
\usepackage{lipsum}%% a garbage package you don't need except to create examples.
\usepackage{fancyhdr}
\pagestyle{fancy}
\lhead{Linear Algebra HW 6}
\rhead{ Page :  \thepage}
\cfoot{Hossein Dehghanipour - 9532250}
% End ofHeader and Footer
\renewcommand{\headrulewidth}{0.4pt}
\renewcommand{\footrulewidth}{0.4pt}
\newcommand{\thedate}{\today}
\setlength{\parindent}{0pt}
\usepackage{atbegshi}% http://ctan.org/pkg/atbegshi
\AtBeginDocument{\AtBeginShipoutNext{\AtBeginShipoutDiscard}}


%====================== Border ================================

% ====================== Begin Doc =============================
\begin{document}

    \begin{titlepage}
        \begin{center}
            \title{\Large{\textbf{Linear Algebra - HW 6}}}
            \author{Hossein Dehghanipour - 9532250}
            \date{\today}
        \end{center}
    \end{titlepage}

    \maketitle
\begin{doublespace}
\begin{flushleft}
\begin{doublespace}
	Teacher :  Dr.Mehrzad Bighash
\end{doublespace} 
\begin{doublespace}
	TA : Ms.F.Lotfi
\end{doublespace} 
\end{flushleft}
\end{doublespace} 

\newpage

\setmainfont{Times New Roman}
\pagenumbering{roman}
\tableofcontents
\thispagestyle{empty}
\newpage
\pagenumbering{arabic}

% *************** Questions Section ******************

\fontsize{15pt}{15pt}
\newcommand{\nnl}{\newline \noindent}


% *************** Define Variables ***********************
\newcommand{\ptx}{ P_{tx} }
\newcommand{\prx}{ P_{rx} }
\newcommand{\inv}[1]{{#1}^{-1}}
\renewcommand\CancelColor{\color{red}}
% *************** Used Commands ***********************
% some multiline commenting using "verbatim" package.

% \begin{comment}
% some comment
% \end{comment}

\begin{comment}
	
	 \left\| 
	 \right\|
	 \prime
	 \pi
	 \underbrace{det(A)*det(A)*....*det(A)}_{\text{n times}}
	  \otimes
	 \ddots : diagonal dots
	  \xrightarrow{\text{*A from left}}
	 \textcolor{blue}{det(AB) = 12 }
	 \Rightarrow
	 \cancel{ det(B)^2}
	 \newcommand{\vfrac}[2]{\ensuremath{\frac{#1}{#2}}}
	
	\begin{doublespace}
	\end{doublespace}
	
	
	 \begin{align*}
		& \\
		& \\
	 \end{align*}
	
	
	
	 \begin{figure}[h!]
		\centering
		\includegraphics*[height=3cm]{PicNameWithOutFileType}
	 \end{figure}


	\[
	    \alpha(x)=\left\{
	                \begin{array}{ll}
	                  x\\
	                  \frac{1}{1+e^{-kx}}\\
	                  \frac{e^x-e^{-x}}{e^x+e^{-x}}
	                \end{array}
	              \right.
	  \]

	\sum_{n=-\infty}^{+\infty} f(x)
	\[ \lim_{x \to 2} f(x) = 5 \]
\end{comment}
%===================================
\section {Question 1}

\begin{figure}[h!]
	\centering
	\includegraphics*[height=3cm]{Q1}
\end{figure}


\begin{doublespace}
As we already know :
$ (A \otimes B ) ( C \otimes D ) = AC \otimes BD $ \\
By using that rule we can say :
$ ( A \otimes B )(\inv{A} \otimes \inv{B} ) = A\inv{A} \otimes B\inv{B} = I  $ \\
We also know that : $ \inv{X}  X  = I \Rightarrow \inv{( A \otimes B )} = (\inv{A} \otimes \inv{B} ) $
\end{doublespace}

\line(1,0){450}
\newpage
%===================================
\section {Question 2 }


\begin{figure}[h!]
	\centering
	\includegraphics*[height=3cm]{Q2}
\end{figure}



\begin{doublespace}
As we know from a theorem in out book : $ \textcolor{purple}{ \inv{ (\inv{A} + \inv{B} )} = A - A\inv{( B + A )} A}$  by which we can conclude that : \\
$ ( X + Y ) = \inv{X} - \inv{X}\inv{(\inv{Y} + \inv{X})}\inv{X} $\\
Now we can check two situations : 

I ) 
	\[ \left\{
	  \begin{array}{lr}
	   X \rightarrow \inv{A}\\
	    Y \rightarrow \inv{B}
	  \end{array}
	\right.
	\]
	\\
	 \begin{align*}
		& \underbrace{\textcolor{brown}{\inv{(\inv{A} + \inv{B})}}}_{RI}   = \textcolor {blue}{A - A\inv{( B + A )}A}
	 \end{align*}

OR

	\[ \left\{
	  \begin{array}{lr}
	   X \rightarrow \inv{B}\\
	    Y \rightarrow \inv{A}
	  \end{array}
	\right.
	\]
	\\
	 \begin{align*}
		& \underbrace{ \textcolor{brown}{\inv{(\inv{B} + \inv{A})} }}_{RII}= \textcolor {blue}{B - B\inv{( A + B )}B}
	 \end{align*}
It's crystal clear that $ \underbrace{ \textcolor{brown}{\inv{(\inv{B} + \inv{A})}}}_{RII} = \underbrace{\textcolor{brown}{\inv{(\inv{A} + \inv{B})}}}_{RI}  $ which would result in : $\\
\textcolor {blue}{A - A\inv{( B + A )}A} =\textcolor {blue}{B - B\inv{( A + B )}B} $ 
\end{doublespace}
\newpage
%===================================

%===================================
\section {Question 3 }

\begin{figure}[h!]
	\centering
	\includegraphics*[height=1cm]{Q3}
\end{figure}

\begin{doublespace}
According to what the question is saying, we should assume that $ \underbrace{(\inv{A} + \inv{B}) }_{X} = \underbrace{\inv{(A + B)}}_{X} $\\
As we already know $( X\inv{X} = I) $ this would mean that : $\underbrace{(\inv{A} + \inv{B}) }_{X}  \underbrace{(A + B)}_{ \inv{X} }     = I $
\end{doublespace}


	 \begin{align*}
		(\inv{A} + \inv{B}) (A + B)  =  & \inv{A}A + \inv{A}B + \inv{B}A  + \inv{B}B  = I \Rightarrow\\
		& \cancel{I} + \inv{A}B + \inv{B}A + I  = \cancel{I} \Rightarrow\\
		&  \inv{A}B + \inv{B}A   = -I\\
	 \end{align*}

\begin{doublespace}

Now we can have two separate approaches :\\
$\textcolor {blue}{I)}$\\
$ \inv{A}B + \inv{B}A   = -I \xrightarrow{\text{*B from left}} B\inv{A}B + A = -B \rightarrow  \textcolor{red}{B\inv{A}B}  = \textcolor{brown}{-A-B} $ \\
$\textcolor {blue}{II)}$\\
$ \inv{A}B + \inv{B}A   = -I \xrightarrow{\text{*A from left}}  B + A\inv{B}A = -A \rightarrow  \textcolor{cyan}{A\inv{B}A}  = \textcolor{brown}{-A-B} $ \\
Now from both (I) and (II) we can conclude that : \\
$ \textcolor{brown}{-A-B} = \textcolor{cyan}{A\inv{B}A} = \textcolor{red}{B\inv{A}B} \Rightarrow  A\inv{B}A = B\inv{A}B $
\end{doublespace}
\line(1,0){450}

\newpage
%===================================
\section {Question 4 }


\begin{figure}[h!]
	\centering
	\includegraphics*[height=3cm]{Q4}
\end{figure}
\begin{doublespace}
As we know from the geometrical series : $S_n = I + A + A^2 + ... + A^n , S_0 = I $\\
\[ S_n  = \Sigma_{n=0}^{n-1} A^n \] \\
We also know that for each eigenvalue of $\lambda_i $, if $|\lambda_i| < 1 $ then the geometric series would converge to :\\
 \[S_n = (I-A^n)(I-A)^{-1}\]\\
 \[ \lim_{x \to \infty } S_n =   I + A + A^2 + ... \]\\
\[ \Sigma_{n=0}^{\infty} A^n =  \lim_{x \to \infty } S_n =  \lim_{x \to \infty} {\inv{(I-A)}(I-A^n)} =   I + A + A^2 + ...  \xrightarrow{\text{$ \lim_{x \to \infty} A^n = 0 $  }}   \] \\
\[   \inv{(I-A)} (I-0) =   \textcolor{red} { \inv{(I-A)} =  I + A + A^2 + ...  }  \] \\

\end{doublespace}

\line(1,0){450}

\newpage
%===================================
\section {Question 5 }

\begin{figure}[h!]
	\centering
	\includegraphics*[height=2cm]{Q5}
\end{figure}

\begin{doublespace}
Assume that we have $Y = AX + X - I \xrightarrow{\text{*A from Right Side}}  YA = A(XA) + XA - A $\\
$  \xrightarrow{\text{( XA = I) Assumed in the problem description }} YA = A + I - A = I \rightarrow YA = I $ \\
In the description of the question, it's mentioned that "X" is the only matrix that satisfies (XA = I). from this hint we can conclude that Y = X . \\
$Y = AX + X - I \xrightarrow{\text{ Y =  X }} X = AX + X - I   \rightarrow  \textcolor{blue}{AX = I}  $
\end{doublespace}
\line(1,0){450}
\newpage
%===================================
End of this HomeWork.\\
This file is written in \LaTeX
% ====================== End Doc =============================
\end{document}